% Exercise 1
\exercise{1}
If $r$ is rational $(r \neq 0)$ and $x$ is irrational, prove that $r + x$ and $rx$ are irrational.

\vspace{2mm}
\solve
We'll start with the proof that $r +x$ is irrational.
After that, we'll move on to the proof that $rx$ is irrational.
Both proofs are very similar.
Note that I will use the notation $r^{-1}$ instead of $1/r$ for multiplicative inverses, since I think the notation is a lot cleaner this way.

\vspace{2mm}
\noindent \underline{$r + x \notin \rat$:}
Assume that $r + x \in \rat$, for the sake of contradiction.
If $r + x$ is rational, then for any $y \in \rat$, $(r + x) + y$ is also rational (since $\rat$ is a field), by the field axiom (A1).
Note that since $r \in \rat$, then $-r \in \rat$ too, by axiom (A5).
This means that
\begin{align*}
    (r + x) + (-r) &\in \rat &\text{by (A1)} \\
    \Implies r + (x + (-r)) &\in \rat &\text{by (A3)} \\
    \Implies r + (-r + x) &\in \rat &\text{by (A2)} \\
    \Implies (r + (-r)) + x &\in \rat &\text{by (A3)} \\
    \Implies 0 + x &\in \rat &\text{by (A5)}\\
    \Implies x &\in \rat &\text{by (A4)}
\end{align*}
By assumption, $x$ is irrational, which means that $x \notin \rat$.
However, we just deduced that $x \in \rat$ from our assumptions.
The only way to resolve this contradiction is to realize that our initial assumption was wrong.
Hence, $r + x \notin \rat$.
In other words, $r + x$ is irrational.

\vspace{2mm}
\noindent \underline{$rx \notin \rat$:}
Similar to the last proof, assume that $rx \in \rat$ for the sake of contradiction.
By axiom (M1), for any $y \in \rat$, it must be true that $(rx)y \in \rat$.
Also, since $r \in \rat$ and $r \neq 0$, it follows that $r^{-1} \in \rat$ too, by axiom (M5).
We can now deduce that
\begin{align*}
    (rx)r^{-1} &\in \rat &\text{by (M1)} \\
    \Implies r (x r^{-1}) &\in \rat &\text{by (M3)}\\
    \Implies r (r^{-1} x) &\in \rat &\text{by (M2)}\\
    \Implies (r r^{-1}) x &\in \rat &\text{by (M3)}\\
    \Implies 1x &\in \rat &\text{by (M5)}\\
    \Implies x &\in \rat &\text{by (M4)}
\end{align*}
You probably get the gist by now, but we cannot have $x \notin \rat$ by assumption and then deduce that $x \in \rat$ by applying the field axioms.
The only possible way to move forward is to conclude that our initial assumption is wrong; 
that is, $rx \notin \rat$.
In other words, $rx$ is irrational.
$\Box$

\newpage
\vspace{5mm}
% Exercise 2
\exercise{2} 
Prove that there is no rational number whose square is 12.


\vspace{2mm}
\solve
The proof is rather similar to the proof that $\sqrt2$ is irrational.
Assume that there is some $x \in \rat$ such that $x^2= 12$.
This means that $x = \frac{m}{n}$, with $m, n \in \bbZ$, such that at most one of $m, n$ is divisible by 3.
Observe that $x^2 = (\frac{m}{n})^2$, meaning that $(\frac{m}{n})^2 = 12$.
All of the above implies that 
$$m^2 = 12n^2.$$
Let's now proceed by checking the different possible cases.

First, suppose that $m$ is not divisible by 3.
Then clearly $m^2$ is not divisible by 3 either.
However, $12n^2$ is definitely divisible by 3.
This would imply that $m^2 \neq 12n^2$, which is clearly not true.
So it can't be true that $m$ is not divisible by 3.

So the only choice we have left is that $m$ \underline{is} divisible by 3.
In this case, $m^2$ must be divisible by 9.
This would imply that $12n^2$ must also be divisible by 9, since $m^2 = 12n^2$, per our work above.
However, this cannot be the case.
12 is not divisible by 9, but it is divisible by 3.
% After dividing, we would have that
% $$\frac{m^2}{3} = 4n^2$$
So we still have a factor of 3 left.
This would imply that $n^2$ must be divisible by 3, but that itself would imply that $n$ is divisible by 3.
However, by assumption, at most one of $m, n$ can be divisible by 3.
This would lead us to conclude that
$m^2 \neq 12n^2,$
a clear contradiction.

We have exhausted all possible choices of $x \in \rat$.
This means that there does not exist any $x \in \rat$ such that $x^2 = 12$.
In other words, there is no rational number whose square is 12.
In other words, 12 does not have a rational square root. 
$\Box$


\newpage
% Exercise 3
\exercise{3}
Prove Proposition 1.15.


\vspace{2mm}
\solve
As a reminder, Proposition 1.15 states the following:

\vspace{2mm}
\noindent\textit{The axioms for multiplication imply the following statements.}
\begin{enumerate}[label={(\alph*)}]
    \item \textit{If} $x \neq 0$ \textit{and}  $xy = xz$ \textit{then} $y = z$.
    \item \textit{If} $x \neq 0$ \textit{and}  $xy = x$ \textit{then} $y = 1$.
    \item \textit{If} $x \neq 0$ \textit{and}  $xy = 1$ \textit{then} $y = x^{-1}$.
    \item \textit{If} $x \neq 0$ \textit{then} $(x^{-1})^{-1} = x$.
\end{enumerate}

Let's prove each statement, one by one.
\begin{enumerate}[label=(\alph*)]
    \item % (a)
    Suppose $x \neq 0$ and $xy = xz$.
    Then
    \begin{align*}
        y &= 1y &\text{by (M4)}\\
        &= (xx^\inv) y &\text{by (M5)}\\
        &= x(x^\inv y) &\text{by (M3)}\\
        &= x(y x^\inv) &\text{by (M2)}\\
        &= (xy)x^\inv &\text{by (M3)}\\
        &= (xz)x^{\inv} &\text{by assumption}\\
        &= (zx)x^\inv &\text{by (M2)}\\
        &= z(xx^\inv) &\text{by (M3)}\\
        &= z1 &\text{by (M5)}\\
        &= z &\text{by (M4)}
    \end{align*}
    So $y=z$.

    
    \item % (b)
    Suppose $x \neq 0$ and $xy = x$.
    Note that $x = x1$, by axiom (M4).
    So it must be true that $xy = x1$.
    Appealing to our work in part (a), we can deduce that $y = 1$.
    
    \item % (c)
    Suppose $x \neq 0$ and $xy = 1$.
    Then
    \begin{align*}
        y &= 1y &\text{by (M4)}\\
        &= (xx^\inv) y &\text{by (M5)}\\
        &= x(x^\inv y) &\text{by (M3)}\\
        &= x(yx^\inv) &\text{by (M2)}\\
        &= (xy)x^\inv &\text{by (M3)}\\
        &= 1x^\inv &\text{by assumption}\\
        &= x^\inv &\text{by (M4)}
    \end{align*}
    So $y = x^\inv$.
    
    \item % (d)
    Assume that $x \neq 0$.
    Then by axiom (M5), $x^\inv x = 1$.
    So by our work in part (c), it must be true that $x = (x^\inv)^\inv$.
\end{enumerate}
And we are done!
$\Box$